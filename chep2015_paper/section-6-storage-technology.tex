\subsection{Storage technology}

The old testbed was based on GridFTP\cite{GFTP1.0}\cite{GFTP2.0} transfers between small storage elements. The software used on the servers were Globus gridftpd, 
DPM (the Disk Pool Manager) and dCache. 
All of these needed configuration changes from the default to work with IPv6.

\begin{itemize}
\item DPM needed a new MySQL (v5.5+) and several configuration changes to do any transfers over IPv6.
\item dCache got GridFTP working in version 2.9.4, but needed no configuration changes.
\item StoRM needs configuration changes but works if it's new enough.
\item XRootD got first IPv6 support in version 4.0.0, it is fully dual-stack since. Version 4.1.0 brought a lot of fixes for the IPv6 functionality.
\item Globus gridftpd needs configuration changes.
\end{itemize}

For the FTS3-based testbed the SRM protocol \cite{SRM2.2} was selected for initiating transfers as it is more real-life production-like 
than the old gridftp-based one. 

All sites participating in the new testbed selected dCache, as it had matured into the only full-stack storage 
system to fully support dualstack and IPv6-only setups. Even with full support for IPv6 as default, the testbeds setup with 
multiple hostnames (dualstack, IPv4-only and IPv6-only) needed configuration changes for the SRM door for it to recognize that it had multiple hostnames.

A new addition was also a IPv6-only storage element, to test how well clients and FTS3 could handle a storage element that is not dual-stacked.
This required very little configuration but would not work without the latest release of dCache, 2.12.



