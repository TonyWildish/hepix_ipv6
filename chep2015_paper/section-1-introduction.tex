The world's Regional Internet Registries are rapidly running out of available IPv4 addresses and the 
general slow transition to IPv6 continues. The Worldwide Large Hadron Collider Grid (WLCG) and the LHC experiments 
may soon have access to worker nodes or virtual machines possessing only an IPv6-routable address. The HEPiX
IPv6 Working Group \cite{ipv6wg} has been investigating the many issues feeding into the move to the use of IPv6 in HEP and WLCG.
The group's paper at CHEP2013 \cite{ipv6chep2013} described the aims of the group and the testing of dual-stack IPv6/IPv4 
services that had been completed at that point. In the last 18 months the group has worked more closely with the
4 major LHC experiments and identified the main use case for the support of IPv6-only clients on WLCG. The groups
activities, including testing of dual-stack data storage services, during the last 18 months are presented in this 
paper together with its future plans.