

In the summer of 2014 a survey of all WLCG sites was performed by the IPv6 working group. This asked a few simple questions 
to determine the site readiness for IPv6 and also whether they foresaw running out of IPv4 address space in the coming years.  

\begin{center}
\begin{table}[h]
\centering
\caption{\label{tsurvey}Site IPv6-readiness}
\begin{tabular}{cccccc}
\br
Type of Site&Answered&IPv6 now&IPv6 soon&No IPv6 plans&Lack of IPv4\\
\mr
Tier 0/1&12&9&3&0&2\\
Tier 2&96&20&20&56&10\\
\br
\end{tabular}
\end{table}
\end{center}

Approximately two-thirds of the WLCG sites responded and the broad conclusions of the survey are summarised in table \ref{tsurvey}.

n.b. numbers in the table still to be checked!


The main conclusions of the survey are that most Tier 1 sites are or will soon be ready, whereas approximately 60\% of the Tier 2 sites have 
not yet started their planning for IPv6. Moreover the fact that about 10\% of the sites foresee problems with the imminent lack of routable
IPv4 addresses means that WLCG must consider moving to the use of dual-stack IPv6/IPv4 services as quickly as possible.


