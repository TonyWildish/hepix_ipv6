\documentclass[a4paper]{jpconf}
\usepackage{graphicx}
\usepackage{color}
\usepackage{array}
\usepackage{enumerate}

\begin{document}
\title{The production deployment of IPv6 on WLCG}

\author{J Bernier$^1$, S Campana$^2$, K Chadwick$^3$, J Chudoba$^4$, 
        A Dewhurst$^5$, M Eli\'a\v s$^4$, S Fayer$^6$, T Finnern$^7$,
        C Grigoras$^2$, B Hoeft$^7$, T Idiculla$^5$, D P Kelsey$^5$,  
        F L\'opez Mu\~noz$^9$, E Macmahon $^{10}$, E Martelli$^2$, R Nandakumar$^5$, 
        K Ohrenberg$^7$, F Prelz$^{11}$, D Rand$^6$, 
        A Sciab\`a$^2$, U Tigerstedt$^{12}$, R Voicu$^{13}$, 
        C J Walker$^{14}$ and T Wildish$^{15}$}

\address{$^1$ IN2P3 Computing Centre, Boulevard du 11 Novembre 1918, F-69622 Villeurbanne Cedex, France}
\address{$^2$ CERN, CH-1211 Gen\`eve 23, Switzerland}
\address{$^3$ Fermi National Accelerator Laboratory, Batavia, Il 60510, U.S.A.}
%\address{$^3$ Institute of High Energy Physics, 19B Yuquanlu, Shijingshan District, 100049 Beijing, China} 
\address{$^4$ Institute of Physics, Academy of Sciences of the Czech Republic Na Slovance 2 182 21 Prague 8, Czech Republic}
\address{$^5$ STFC Rutherford Appleton Laboratory, Harwell Oxford, Didcot, Oxfordshire OX11 0QX, United Kingdom}
\address{$^6$ Imperial College London, South Kensington Campus, London SW7 2AZ, United Kingdom}
\address{$^7$ Deutsches Elektronen-Synchrotron, Notkestra\ss e 85, D-22607 Hamburg, Germany}
\address{$^8$ Karlsruher Institut f\"ur Technologie, Hermann-von-Helmholtz-Platz 1, D-76344 Eggenstein-Leopoldshafen, Germany}
\address{$^9$ Port d'Informaci\'o Cient\'ifica, Campus UAB, Edifici D, E-08193 Bellaterra, Spain}
\address{$^{10}$ The University of Oxford, Denys Wilkinson Building, Keble Road, Oxford OX1 3RH, United Kingdom}
\address{$^{11}$ INFN, Sezione di Milano, via G. Celoria 16, I-20133 Milano, Italy}
\address{$^{12}$ CSC Tieteen Tietotekniikan Keskus Oy, P.O. Box 405, FI-02101 Espoo}
\address{$^{13}$ California Institute of Technology, Pasadena, Ca 91125, U.S.A.}
\address{$^{14}$ Queen Mary University of London, Mile End Road, London E1 4NS, United Kingdom}
\address{$^{15}$ Princeton University, Jadwin Hall, Princeton, NJ 08544, U.S.A.}

\ead{david.kelsey@stfc.ac.uk, ipv6@hepix.org}

\begin{abstract}
The world is rapidly running out of IPv4 addresses; the number of IPv6 end systems connected
to the internet is increasing; WLCG and the LHC experiments may soon have access to worker
nodes and/or virtual machines (VMs) possessing only an IPv6 routable address. The HEPiX
IPv6 Working Group ({\tt http://hepix-ipv6.web.cern.ch/}) has been investigating, testing and
planning for dual-stack services on WLCG for several years. Following feedback from our
working group, many of the storage technologies in use on WLCG have recently been made
IPv6-capable. The worldwide HEP computing community now needs to deploy dual-stack
IPv6/IPv4 services on WLCG to allow such use of IPv6-only resources.
This paper will present the IPv6 requirements, tests and plans of each of the four LHC
experiments together with the tests performed both on the IPv6 test-bed and in targeted use
of WLCG production services. This is primarily aimed at IPv6-only worker nodes or VMs
accessing several different implementations of a global dual-stack federated storage service.
The changes required to the operational infrastructure, including monitoring and security, will
be addressed as will the implications for site management. The working group will present
its deployment plan for dual-stack storage services, together with other essential central and
monitoring services, to start during 2015.
\end{abstract}

\section{Introduction}
blah blah


% Add other sections as appropriate...

\par
\section*{References}

\begin{thebibliography}{1}
\bibitem{ipv6wg} {\tt http://hepix-ipv6.web.cern.ch}
\bibitem{rfc} All Internet Engineering Task Force Requests For Comments (RFC) documents are available
from URLs such as http://www.ietf.org/rfc/rfcNNNN.txt where NNNN is the RFC number, for example {\tt http://www.ietf.org/rfc/rfc2460.txt}
\bibitem{ipv6stat} See for instance {\tt http://www.google.com/ipv6/statistics.html}. The 2\% global connectivity threshold was crossed in September 2013.
\bibitem{rhel} {\tt http://www.redhat.com/products/enterprise-linux/}
\bibitem{cream}
Aiftimiei C, Andreetto P, Bertocco S, Dalla Fina S, Dorigo A,
Frizziero E, Gianelle A, Marzolla M, Mazzucato M, Sgaravatto M,
Traldi S, Zangrando L 2010 Design and Implementation of the gLite CREAM Job
Management Service {\it Future Generation Computer Systems} Volume {\bf 26} Issue
4 pp 654-667, doi: 10.1016/j.future.2009.12.006.
\bibitem{wms}
Cecchi M, Capannini F, Dorigo A, Ghiselli A, Giacomini F, Maraschini M, Marzolla M, Monforte S, Pacini F, Petronzio L, Prelz F 2009 The gLite Workload Management System {\it Advances in Grid and Pervasive Computing: 4th International Conference, GPC}
\bibitem{panda}
Maeno T 2008 PanDA: distributed production and distributed analysis
system for ATLAS {\it J. Phys. Conf. Ser.} {\bf 119} 062036
\bibitem{fts}
Kunszt P, Badino P, Rocha R, Casey J, Frohner A, McCance G 2006 The gLite File Transfer Service
{\it Workshop on Next Generation Distributed Data Management at The Fifteenth IEEE International Symposium on High-Performance Distributed Computing (HPDC2006), Paris, France}
\bibitem{phedgen}
Egeland R, Metson S and  Wildish T 2008 Data transfer infrastructure for
CMS data taking,  {\it XII Advanced Computing and Analysis Techniques in
Physics Research (Erice, Italy: Proceedings of Science)}
\bibitem{cvmfs}
Blomer J et al 2012 Status and future perspectives of CernVM-FS
{\it J. Phys.: Conf. Ser.} {\bf}396 052013
\bibitem{bdii}
Field L and Schulz M W 2004  Grid Deployment Experiences: The path to a production quality LDAP based grid information system {\it Proceedings of the International Conference on Computing in High Energy and Nuclear Physics (CHEP 2004)}
\bibitem{monalisa}
Legrand I, Newman H, Voicu R, Cirstoiu C, Grigoras C, Dobre C, Muraru A,
Costan A, Dediu M and Stratan C 2009 MonALISA: An agent based, dynamic
service system to monitor, control and optimize distributed systems {\it
Computer Physics Communications} Volume {\bf180} Issue 12, December 2009,
Pages 2472–2498
\bibitem{LifeCycle}
    Wildish T 2013 Integration and validation testing for PhEDEx, DBS and DAS with the PhEDEx LifeCycle agent {\it also presented at CHEP 2013}
\bibitem{cms}
The CMS Collaboration 2008 The CMS experiment at the CERN LHC {\it JINST
{\bf 3} S08004}
\bibitem{PhEDEx}
    Egeland R, Wildish T, Metson S 2008 Data transfer infrastructure for CMS data taking {\it XII Advanced Computing and Analysis Techniques in Physics Research (Erice, Italy: Proceedings of Science)}
\bibitem{FTS3}
    Salichos M, Keeble O, Alvarez Ayllon A, Kamil Simon M 2013 FTS3 - Robust, simplified and high-performance data movement service for WLCG {\it also presented at CHEP 2013}
\end{thebibliography}


\end{document}

