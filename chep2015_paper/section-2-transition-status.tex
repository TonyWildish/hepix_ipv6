We now offer a quick panorama of the IPv6 statistical trends since the last
CHEP conference and identify two factors that may be currently limiting 
the IPv6 adoption rate. 
\subsection{Survey of available statistical data}
An analysis of the available statistical data collected since 2013
by the Regional Internet Registries \cite{ripeipv6,arinstat,apnicstat,afrinicipv6,lacnicipc6} and by major internet
service providers \cite{akamaistats,googlestats} shows a steady, but
still polynomial growth in the volume of IPv6 traffic from roughly 
2\% up to 6\% of the total. 
\begin{figure}
\centering
\def\svgwidth{\columnwidth}
\input{bgp_updates.pdf_tex}
\caption{Number of routing topology updates involving the reference LIR of the working group testbed participants recorded by RIPEstat \cite{ripestat} every 12 hours since CHEP2013.}
\label{fig:bgpupdates}
\end{figure}
Not everything has been progressing at a relaxed pace, though: we collected from RIPEstat \cite{ripestat} into Figure \ref{fig:bgpupdates} the
rate of BGP routing topology updates that affects
the IPv6 {\tt/32} prefixes that serve our
working group testbed. We notice a significantly increased activity rate over the
last year. We can probably read this as the effect of organising and troubleshooting efficient, production-proof routing for our community.\par
On the IPv4 address exhaustion side, the actual availability of
assignable IPv4 addresses has remained substantially stable around the 18 million mark at RIPE (Europe). AFRINIC (Africa) still has all the 16 million addresses in the last assigned {\tt /8} network available, APNIC (Asia-Pacific) has 12 million address left, ARIN (North America) has 4 million and LACNIC (South America) is in
the most critical state with 3 million IPv4 addresses left.
All Regional Internet Registries are
now implementing IPv4 request limits and 'soft landing' allocation schemes.
\subsection{Slow IPv6 adoption progress rate: why?}
While it is in the best interest of a successful transition to IPv6 that
no show-stoppers be found and the process keeps moving {\it forward},
one may wonder about the causes that prevent a more rapid adoption pace.\par
Transport and provider issues have to be excluded right away: most
local registries, including all of the national research networks, have been
providing IPv6 transport for 7 years or more \cite{ripeness}. Performance issues also have to
be ruled out: available studies, especially the ones carried out during the
``world IPv6 launch'' in 2012 \cite{wdayperf} show performance on 
the two stacks to be comparable within statistics. The reality of the IPv4 
address depletion
should also be widely perceived by now, as many regional registries are
handling the {\it final} IPv4 assignment to local registries.\par
Two residual classes of factors may be quenching IPv6 adoption, one
affecting network administrators and one affecting application developers:
\begin{enumerate}
\item {\small
The IPv4/v6 difference in address allocation schemes, the pivot role of
ICMPv6 Router Advertisements, the short-term
need to implement measures to counter rogue Router Advertisements 
(see RFC6104, \cite{rfc}) are all adding to the already sizeable initial 
investment of implementing monitoring and security tools for IPv6.\\
Also, existing IPv6 code in the Operating Systems and related tools often
shows by inspection not to have undergone full coverage testing:
a phase of initial fault finding and patching is foreseen and feared.
}
\item {\small
Apart from the syntactical differences in IPv6 addresses (e.g. parsing
'{\tt defa}' in {\it default} as a hex digit), and the need to label, sort and 
pick IPv4 vs IPv6 addresses, a large {\it semantical} change is
needed in applications supporting IPv6. Every network endpoint on the
public IPv6 network has {\it at least} two IPv6 addresses assigned (global
and link-local), and possibly more. Applications have therefore
to {\it always} deal with multi-homed network endpoints, which means a complex
$1\rightarrow n$ change for many of them. The status of porting to IPv6 of
many applications of interest for our community is good \cite{readiness},
but the related development effort cannot be underestimated.
}
\end{enumerate}
