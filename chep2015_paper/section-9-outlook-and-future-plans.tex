The HEPiX IPv6 working group has made good progress during the last 18 months. It has been demonstrated that access 
to remote dual-stack federated data storage services in a production-like environment functions well with FTS, SRM and dCache.
During the remainder of 2015 tests on other storage technologies will be performed. Members of the group will deploy dual-stack services
on more production instances of storage services and other essential central services to enable the proper measurement of 
data transfer performance over IPv6 and to 
demonstrate to the experiments and the WLCG management that it is safe to migrate to dual-stack. The deployment of IPv6 peering on 
LHCOPN/LHCONE and dual-stack perfSONAR instances will be tracked and encouraged. Both of these are pre-requisites to the wider deployment
of production dual-stack services. In parallel with these activities the group also aims to provide more training sessions and guidance
on best practice in the management of IPv6 services and site operations. Once there are a sufficient number of dual-stack services 
deployed on WLCG it will be possible to support the use of IPv6-only clients within the production infrastructure.