%\section{IPv6 perfSONAR measurements}

The WLCG has adopted the perfSONAR toolkit \cite{perfsonar} for
the monitoring of its network infrastructure and this project is
being coordinated by the WLCG Network and Transfer Metrics group
\cite{wlcg-NTMG}.  The WLCG perfSONAR configuration system operates around
groups of sites with a common purpose and these are known as meshes.
For example, there are meshes for each WLCG country
group (e.g. UK, DE, FR etc.) or experiment such as USATLAS or USCMS or
network groupings such as LHCONE and LHCOPN. Until recently testing
between members of the groups was configured using JSON files held on a
web-server at CERN specifying the group members and test parameters. A
site administrator configuring a perfSONAR host for a mesh needed to add
the URL of the JSON file corresponding to that mesh into a configuration
file on the perfSONAR host. This worked but required effort from all
site administrators involved. The mesh configuration system has recently
undergone development. This is described in detail in \cite{wlcg-NTMG},
but briefly, the system has evolved from the use of the manually
configured JSON files to a more automated system which is significantly
easier to configure. Each perfSONAR host now has a so-called auto-mesh
URL e.g. \\{\tt\small  
https://myosg.grid.iu.edu/pfmesh/mine/hostname/psum01.aglt2.org}\\
containing configuration details for the meshes that the perfSONAR host
has been added to. The meshes are maintained using a mesh-configuration
GUI provided by the US Open Sciences Grid (OSG) using data collected
from both the GOCDB and OSG Information Management System (OIM). The
perfSONAR toolkit has the ability to monitor both IPv4 and IPv6 network
connectivity. Consequently, in addition to the meshes mentioned above, we
have added a mesh containing perfSONAR hosts known to have both IPv4 and
IPv6 connectivity, i.e. a dual-stack mesh. The mesh tests throughput and
latency between hosts over both IPv4 and IPv6. Results are available from
the web sites of the relevant perfSONAR hosts, for example the perfSONAR
bandwidth host at WLCG site UKI-SOUTHGRID-OX-HEP at the University of
Oxford: \\{\tt\small http://t2ps-bandwidth.physics.ox.ac.uk/toolkit/}\\ 
