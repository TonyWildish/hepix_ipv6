The HEPiX forum brings together worldwide IT staff, including system administrators, system engineers, and managers from the High Energy Physics and Nuclear Physics laboratories and institutes, to foster a learning and sharing experience between sites facing scientific computing and data challenges. At its semi-annual meetings, HEPiX had been considering the issue of migration to IPv6 for a number of years. In 2011 a survey of HEP sites around the world was made asking about their plans for the deployment of IPv6. While it was very clear that there was no requirement for an urgent move to IPv6 a good number of sites were planning such a deployment and a few, particularly CERN, reported a foreseen lack of IPv4 address space in the not too distant future.

It was realised that any decision to deploy IPv6 on the WLCG infrastructure would involve much testing and planning and the decision was therefore taken to start a dedicated working group to investigate the issues. The HEPiX IPv6 working group was formed in 2011 with the following mandate.

Phase 1 of the work was to consider whether and how IPv6 should be deployed in HEP (especially for WLCG). This involved:
\begin {itemize}
\item A Readiness and Gap analysis
\item The need to include all relevant HEP applications, middleware, security issues, system management and monitoring tools, and end-to-end network monitoring
\item Running a distributed HEPiX IPv6 testbed to explore all of the above issues
\item An initial report at the end of 2011
\end {itemize}	
Following that initial report it was agreed that the work should continue and that eventually phase 2 of the work should include:
\begin {itemize}
\item The proposal of a timetable and an analysis of the resources required for the deployment of IPv6 on WLCG
\item The production of an implementation plan including advice to HEP sites on deployment
\end {itemize}

Since then the group has been working on the mandated tasks and meeting regularly with quarterly face to face meetings at CERN and monthly video/phone meetings to review progress. 

Full details of the meetings are available at {\tt http://indico.cern.ch/categoryDisplay.py?categId=3538}


