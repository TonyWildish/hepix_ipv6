\documentclass[a4paper]{jpconf}
\usepackage{graphicx}
\usepackage{color}
\usepackage{array}
\usepackage{enumerate}

\begin{document}
\title{WLCG and IPv6 - the HEPiX IPv6 working group}

\author{K. Chadwick$^1$, G. Chen$^2$, J. Chudoba$^3$, P. Clarke$^4$, M. Elias$^3$,
        A. Elwell$^5$, S. Fayer$^6$, Q. Fazhi$^2$, T. Finnern$^7$, L. Goossens$^5$,
        C. Grigoras$^5$, B. Hoeft$^8$, D. Kelsey $^9$, T. Kouba$^3$, 
        F. Lopez Mu\~noz$^{10}$, E. Martelli$^5$, M. Mitchell$^{11}$, A. Nairz$^5$, 
        K. Ohrenberg$^7$, A. Pfeiffer$^5$, F. Prelz$^{12}$, D. Rand$^6$, 
        M. Reale$^{13}$, S. Rozsa$^{14}$, R. Voicu$^{14}$, T. Wildish$^{15}$}

\address{$^1$ Fermi National Accelerator Laboratory, Batavia, Il 60510, U.S.A.}
\address{$^2$ Institute of High Energy Physics, 19B YuquanLu, Shijingshan District, 100049 Beijing, China} 
\address{$^3$ Institute of Physics, Academy of Sciences of the Czech Republic Na Slovance 2 182 21 Prague 8, Czech Republic}
\address{$^4$ The University of Edinburgh, Mayfield Road, Edinburgh EH9 3JZ, United Kingdom}
\address{$^5$ CERN, CH-1211 Gen\`eve 23, Switzerland}
\address{$^6$ Imperial College London, South Kensington Campus, London SW7 2AZ, United Kingdom}
\address{$^7$ Deutsches Elektronen-Synchrotron, Notkestra\ss e 85, D-22607 Hamburg, Germany}
\address{$^8$ Karlsruher Institut f\"ur Technologie, Hermann-von-Helmholtz-Platz 1, D-76344 Eggenstein-Leopoldshafen, Germany}
\address{$^9$ STFC Rutherford Appleton Laboratory, Harwell Oxford, Didcot, Oxfordshire OX11 0QX, United Kingdom}
\address{$^{10}$ Port d'Informaci\'o Cient\'ifica, Campus UAB, Edifici D, E-08193 Bellaterra, Spain}
\address{$^{11}$ University of Glasgow, Kelvin Building, University Avenue, Glasgow, G12 8QQ, United Kingdom}
\address{$^{12}$ INFN, Sezione di Milano, via G. Celoria 16, I-20133 Milano, Italy}
\address{$^{13}$ Consortium GARR, Via dei Tizii 6, I-00185 Roma, Italy}
\address{$^{14}$ California Institute of Technology, Pasadena, Ca 91125, U.S.A. }
\address{$^{15}$ Princeton University, Jadwin Hall, Princeton, NJ 08544, U.S.A.}

\ead{ipv6@hepix.org}

\begin{abstract}
The HEPiX ({\tt http://www.hepix.org}) IPv6 Working Group has 
been investigating the many issues which feed into the decision on the 
timetable for the use of IPv6 networking protocols in HEP Computing, in 
particular in WLCG. RIPE NCC, the European Regional Internet Registry, ran 
out of IPv4 addresses in September 2012. The North and South America RIRs 
are expected to run out in 2014. In recent months it has become more clear 
that some WLCG sites, including CERN, are running short of IPv4 address space, 
now without the possibility of applying for more. This has increased the 
urgency for the switch-on of dual-stack IPv4/IPv6 on all outward facing WLCG 
services to allow for the eventual support of IPv6-only clients. The 
activities of the group include the analysis and testing of the readiness 
for IPv6 and the performance of many required components, including the 
applications, middleware, management and monitoring tools essential for 
HEP computing. Many WLCG Tier 1 and Tier 2 sites are participants in the 
group's distributed IPv6 testbed and the major LHC experiment collaborations 
are fully engaged in the testing. We have worked closely with similar 
activities elsewhere, such as EGI and EMI. We are constructing a group 
web/wiki which will contain useful information for sites on the IPv6 
readiness of the various software components. This includes advice on 
IPv6 configuration and deployment issues for sites 
({\tt http://hepix-ipv6.web.cern.ch/knowledge-base}). \\
This paper will describe the work done by the HEPiX IPv6 working group 
since CHEP2012. This will include detailed reports on the testing of 
various WLCG services on IPv6 including data management, data transfer, 
workload management and system/network monitoring. It will also present 
the up to date list of those applications and services which function 
correctly in a dual-stack environment together with those that still 
have open issues. The plan for more testing on the production infrastructure 
with a dual-stack IPv4/IPv6 setup and the work required before the support 
of IPv6-only clients is realised will be described.
\end{abstract}

\section{Introduction}
blah blah


\section{IPv6: the general problem}
\section{IPv6: the general problem}

The intention of the designers of the IPv6 protocol was to make it full of appealing features, in order to push its adoption widely and quickly. The IPv6 specifications (RFC 1883) were set back in the 1995, when the Internet community realized that the classful allocation policies of the time were causing a quick depletion of the address space. 
\par
Unfortunately for IPv6, the IPv4 problem was quickly fixed with the adoption of the classless allocations (CIDR, RFC 1519) and by the invention of Address and Port translation techniques (NAT, RFC 1631). These events, together with the fact that the IPv6 advantages were far less appealing than the cost of deploying it, put the protocol in a limbo where it stayed for almost twenty years, until IPv4 addresses became scarce again.
\par
At the end of the first decade of the 21st century, Regional Internet Registries started warning the Internet community that IPv4 addresses were soon be exhausted and urged everyone to adopt IPv6. IPv6 was quite quickly deployed on the Internet backbones, but not where it would have brought the most of its benefits, at the client and content side. 
Plagued by the chicken and the egg problem (no users if no content, no content if no users), in 2012 finally some of the 
biggest content providers made the bold move to make their services available over IPv6. 
One year later, the IPv6 global traffic is gradually increasing but still counts as a very small fraction of the 
total Internet traffic \cite{ipv6stat}.

\section{IPv6 at CERN}
Many academic institutions, which joined the Internet when it was in a very early stage, are still enjoying the large allocations that were given in those days; thus they are lacking of any urge to move to IPv6.
\par 
This was the situation at CERN till 2011, when Server Virtualization started being used. The virtualization technique proved to be very effective and its adoption at CERN has grown exponentially. In 2012, when the plan for the services to be run in the upcoming remote data centre in Wigner (Budapest, HU) was finalized, it became clear that something like 250,000 public IP addresses would be needed in the near future. At the same time, RIPE, the European Internet Registry, was announcing the adoption of a new conservative allocation policy that would grant no more than 1024 IPv4 addresses to any requester. 
It could have been an impasse for the IT deployment plans, but luckily CERN had been testing with IPv6 since 1998 and in 2011 the management of the IT department approved the project to deploy IPv6 in the CERN campus and datacentres, when its need had yet to be proven.
\par
For large enterprises like CERN, deploying IPv6 is not as simple as configuring a dozen of routers to be dual stack. The Network Management System and the Network database had to be made IPv6 aware, all the IPv6 information generated, all the basic network services configured (DNS, NTP, DHCPv6...). At the same time the network security had to be kept at the same level as always.
After two years, the deployment is almost completed. Right in time to tackle the IPv4 exhaustion problem that most likely will hit CERN in 2015 when the Wigner datacentre will reach its full capacity. 
Many applications still cannot make use of IPv6, thus it is very premature to deploy IPv6-only Virtual Servers. The strategy will be to deploy a hybrid solution where servers get a private IPv4 address and a public IPv6 one. The private IPv4 address will allow legacy applications to work within the CERN domain,  while the public IPv6 address will allow world-wide reachability.
\par
Hopefully the availability of LHC data over IPv6 will push IPv6 adoption in the large WLCG community.


\section{The HEPiX IPv6 working group}
The HEPiX forum brings together worldwide IT staff, including system administrators, system engineers, and managers from the HEP and Nuclear Physics laboratories and institutes, to foster a learning and sharing experience between sites facing scientific computing and data challenges. At its semi-annual meetings, HEPiX had been considering the issue of migration to IPv6 for a number of years. At the end of 2010 a survey of HEP sites around the world was made asking about their plans for the deployment of IPv6. While it was very clear that there was no requirement for an urgent move to IPv6 a good number of sites were planning such a deployment and a few, particularly CERN, reported a foreseen lack of IPv4 address space in the not too distant future.

It was realised that any decision to deploy IPv6 on the WLCG infrastructure would involve much testing and planning and the decision was therefore taken to start a dedicated working group to investigate the issues. The HEPiX IPv6 working group was formed in 2011 with the following mandate.

Phase 1 of the work was to consider whether and how IPv6 should be deployed in HEP (especially for WLCG). This involved:
\begin {itemize}
\item A Readiness and Gap analysis
\item The need to include all relevant HEP applications, middleware, security issues, system management and monitoring tools, and end-to-end network monitoring
\item Running a distributed HEPiX IPv6 testbed to explore all of the above issues
\item An initial report at the end of 2011
\end {itemize}	
Following that initial report it was agreed that the work should continue and that phase 2 of the work should include:
\begin {itemize}
\item The proposal of a timetable and an analysis of the resources required for the deployment of IPv6 on WLCG
\item The production of an implementation plan including advice to HEP sites on deployment
\end {itemize}

Since then the group has been working on the mandated tasks and meeting regularly with quarterly face to face meetings at CERN and monthly video/phone meetings to review progress. Full details are available at {\tt http://indico.cern.ch/categoryDisplay.py?categId=3538}.




\section{The IPv6 testbed}
The impact of IPv6 is not limited to the transport layer 
but introduces the need for choice and preference in name-to-address 
resolution, implies multi-homing of all network endpoints (possibly on 
multiple protocol versions) and requires opaque handling of address 
information. This broadens the scope of code changes needed to add
IPv6 support to existing code and adds to the complexity of testing:
continued operation on IPv4 on dual-stack hosts, then preference of IPv6 and 
options to control it for all network bindings and connections
need to be verified with adequate code coverage.
\par
At opposite ends of the spectrum of practical testing options are
testing of individual, isolated components and services and the analysis of 
integrated services on dual-stack nodes. Both approaches are incomplete:
\begin{itemize}
\item[-] testing of isolated components misses the interaction with other
services at the OS level and usually requires services to be configured
differently than for production;
\item[-] testing of production-ready, integrated nodes may just be 
accidentally focusing on normal operation and bring insufficient 
code and functionality coverage.
\end{itemize}
This calls for a complementary approach,
where individual services are deployed and tested within the scale of
available dedicated resources and, once sufficient confidence and knowledge
of their level and mode of IPv6 support is built, are watched in the context
of a production node with dual-stack network and dual IPv4/IPv6 public address
resolution.
\par
Desirable characteristics of a dedicated testbed for single-service testing are:
\begin{itemize}
\item Geographical spread covering all ranges of realistic network latencies
and as many network providers as possible.
\item Uniform authentication/authorization scheme to factor out AA issues.
\item Uniform OS installation, to factor out any issue with the custom 
configuration needed to test isolated services and for easier
deployment of new services.
\end{itemize}
The current list of active testbed nodes can be found at the following URL:\\
{\tt http://hepix-ipv6.web.cern.ch/testbed-nodes}\\
While we have at the time of writing a reasonable
9-site/6-National Research and Education Network (NREN) coverage of Europe, the only non-european sites in the testbed
are IHEP Beijing in China and Fermilab in the US. More testbed sites are both needed and welcome to join 
to achieve a better match of our stated testbed goals.
\par
As for OS distributions, testbed nodes are mostly installed with 
Scientific Linux (CERN) version 5,
to replicate production conditions at LCG Tier-X centres. A few testbed nodes
have Red Hat Enterprise Linux (RHEL) \cite{rhel} version 6 derivatives installed: this allowed us to discover 
(and document in our knowledge base, {\tt http://hepix-ipv6.web.cern.ch/knowledge-base})
that, rather unexpectedly, {\tt libc} on RH6 causes unspecified 
protocol sockets to be bound on IPv4 only, instead of dual-stack 
as it used to be.
\par
To achieve the simplest possible authentication scheme, a custom Globus
Security Infrastructure (GSI)
plug-in maps all members of our test Virtual Organisation (VO) ({\tt ipv6.hepix.org}) to one
local account, logging any access. 
\par
The first service we deployed for standalone testing through the testbed was
GridFTP (or, more accurately, {\tt gsiftp}). 
This was not only because of GridFTP's basic role in WLCG data
transfer, but also because the FTP protocol (the GSI extensions don't 
affect but also suffer from this issue) is a paradigmatic example of 
how non-trivial IPv6 support can be. The original FTP specifications 
(RFC 765/RFC 959 \cite{rfc}) used the quad-byte notation for IP addresses 
{\em in the syntax of the FTP protocol commands} {\tt PORT} and {\tt PASV}.
This required the introduction, with RFC 2428 (September 1998) \cite{rfc} of
``extended'' versions of the same commands, supporting different address
families, and IPv6 in particular. We found on our testbed that support for 
the 'extended' command forms (and thus {\em implicitly} for IPv6)
is missing from certain FTP client implementations. Retrofitting the
clients with these commands is definitely more than a simple change in
the transport layer and serves as an example of how ramified the
introduction of 'IPv6 support' can be.
\par
Building on this mesh of GridFTP servers, both continuous direct point-to-point 
file transfer tests and tests of the File Transfer Service 
(FTS), were successfully carried out. Storage Resource Manager (SRM) endpoints
were also added, as described in detail in the next section.


\section{IPv6 testing and results}
\subsection{Testing testing...}
This is where the testing results go


\section{Software and tools survey}
For a successful transition to the use of IPv6 it is necessary to do a full survey of all important applications, middleware and operational tools. We decided to focus on the IPv6 readiness of the important WLCG outward-facing services and all essential applications and management and monitoring tools. We have created an online database of IPv6-readiness where for each software component considered we can store its known state of readiness and details of any testing performed by us or others.

IPv6-readiness is not a simple yes/no question. There are various stages of readiness to be addressed:

	1.	Does the service break/slow down when used with IPv4 on a dual-stack host with IPv6 enabled ?
	2.	Will the service try using (connecting/binding to) an IPv6 address (AAAA record), when available from DNS ?
	3.	Will the service prefer IPv6 addresses from DNS, when preferred at the host level ? 
Does this need to be configured ? How ?
	4.	Can the service be persuaded to fall back on IPv4 if needed ?

In many ways the most important question is the first one. As long as a service deployed on a dual-stack host behaves properly for IPv4 then we are safe to recommend such a deployment on the WLCG production infrastructure.

The current state of our survey may be seen at {\tt http://hepix-ipv6.web.cern.ch/wlcg-applications}.

At the time of writing this paper, there are still many packages which need further investigation and testing.
Software known to be not ready for IPv6 at this time includes OpenAFS servers and clients, all but the latest release of dCache and many batch systems. Full details of all such problems and investigations will be recorded in our online database.




\section{Collaboration with other groups}
blah blah


\section{Outlook and future plans}
blah blah


\section{Conclusions}
blah blah


\par
\section*{References}

\begin{thebibliography}{1}
\bibitem{ciscosec} S. Hogg, E. Vyncke,
    {\it ``IPv6 security''}, Cisco Press 2008 ISBN-13: 978-1-58705-594-2
\end{thebibliography}

\end{document}

