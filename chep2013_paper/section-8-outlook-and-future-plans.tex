The IPv6 working group has made good progress but has to date only tested a small fraction of the many 
software components required by WLCG. There are still many more tests and 
assessments to be made and advice on dual-stack deployment to be formulated before 
our work can in any way be defined as complete.

Testing will continue in three areas. Firstly we will continue the mesh of data transfer tests between
the testbed sites, expanding to include new types of storage element and allowing for the ongoing
assessment of reliability. Some testbed sites are now working on the deployment of larger scale IPv6 testbeds
to allow the testing of IPv6-readiness in a more realistic production environment. Finally, we will encourage 
more Tier 2 sites to repeat the
tests at Imperial College (see section 6.4) by enabling dual-stack services on some or all of their production
services. Across all of these different scenarios we will gradually work through the list of testing scenarios presented
in section 6.1 together with representatives of the experiments.  During all of this testing we
will continue to update our online database with the details of IPv6 readiness.

At the time of our status report to HEPiX and the WLCG Management Board in 2012 we concluded that
the support of IPv6-only clients on WLCG was unlikely to be possible before January 2014. At the time of writing this is still true and indeed it is now clear that it will take much longer. Not only does the working group still have many tests to perform
but all of the many Tier 1 and Tier 2 sites need to complete the deployment of an IPv6 infrastructure at their site. This needs to include
the revision of local procedures and management tools and the provision of adequate training for network,
system operations and security staff. Once we have achieved this we can propose a more general deployment of 
production-level dual stack services thereby allowing for the eventual support of IPv6-only clients on WLCG.
