The much-heralded exhaustion of the IPv4 networking address space has recently started,
but while the backbone networks are fully ready to support IPv6 there is still a well-known lack of communities and institutes using IPv6.
Many of the individual components of distributed applications have already been reported to be IPv6-ready,
but to ensure a smooth transition by the HEP community a full-systems analysis 
is required. This is a time consuming and complicated endeavour.
The HEPiX IPv6 Working Group (ref) has been investigating the many issues feeding into the 
transition to the use of IPv6 on the Worldwide Large Hadron Collider Grid (WLCG). 
Its activities include the analysis and testing of the readiness for IPv6 and 
performance of the many different components essential for HEP computing and planning for the impact on operations and security. The work of the group to date is presented in this paper together with its future plans.
