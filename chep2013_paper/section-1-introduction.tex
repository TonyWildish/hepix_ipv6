The much-heralded exhaustion of the IPv4 networking address space has finally started,
but while the backbone networks have been ready for many years to support IPv6 there is still a well-known lack of 
applications using IPv6.
Many of the individual components of distributed applications have already been reported to be IPv6-ready,
but to ensure a smooth transition for the applications of interest to the HEP community a full-systems analysis 
is required. This is a time consuming and complicated endeavour.
The HEPiX IPv6 Working Group ({\tt http://hepix-ipv6.web.cern.ch/}) was formed to investigate the many issues feeding into the 
transition to the use of IPv6 in HEP and in particular by the Worldwide Large Hadron Collider Grid (WLCG). 
The group's activities include the analysis and testing of the readiness for IPv6 and 
performance of the many different components essential for WLCG and planning for the impact on 
operations and security. The work of the group to date is presented in this paper together with its future plans.
